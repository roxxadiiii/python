
\documentclass[a4paper,11pt]{article}

\usepackage[T1]{fontenc}
\usepackage[utf8]{inputenc}
\usepackage{graphicx}
\usepackage{xcolor}

\renewcommand\familydefault{\sfdefault}
\usepackage{tgheros}
\usepackage[defaultmono]{droidsansmono}

\usepackage{amsmath,amssymb,amsthm,textcomp}
\usepackage{enumerate}
\usepackage{multicol}
\usepackage{tikz}

\usepackage{geometry}
\geometry{left=25mm,right=25mm,%
bindingoffset=0mm, top=20mm,bottom=20mm}


\linespread{1.3}

\newcommand{\linia}{\rule{\linewidth}{0.5pt}}

% custom theorems if needed
\newtheoremstyle{mytheor}
    {1ex}{1ex}{\normalfont}{0pt}{\scshape}{.}{1ex}
    {{\thmname{#1 }}{\thmnumber{#2}}{\thmnote{ (#3)}}}

\theoremstyle{mytheor}
\newtheorem{defi}{Definition}

% my own titles
\makeatletter
\renewcommand{\maketitle}{
\begin{center}
\vspace{2ex}
{\huge \textsc{\@title}}
\vspace{1ex}
\\
\linia\\
\@author \hfill \@date
\vspace{4ex}
\end{center}
}
\makeatother
%%%

% custom footers and headers
\usepackage{fancyhdr}
\pagestyle{fancy}
\lhead{}
\chead{}
\rhead{}
%\lfoot{Assignment \textnumero{} 5}
\cfoot{}
%\rfoot{Page \thepage}
\renewcommand{\headrulewidth}{0pt}
\renewcommand{\footrulewidth}{0pt}
%

% code listing settings
\usepackage{listings}
\lstset{
    language=Python,
    basicstyle=\ttfamily\small,
    aboveskip={1.0\baselineskip},
    belowskip={1.0\baselineskip},
    columns=fixed,
    extendedchars=true,
    breaklines=true,
    tabsize=4,
    prebreak=\raisebox{0ex}[0ex][0ex]{\ensuremath{\hookleftarrow}},
    frame=lines,
    showtabs=false,
    showspaces=false,
    showstringspaces=false,
    keywordstyle=\color[rgb]{0.627,0.126,0.941},
    commentstyle=\color[rgb]{0.133,0.545,0.133},
    stringstyle=\color[rgb]{01,0,0},
    numbers=left,
    numberstyle=\small,
    stepnumber=1,
    numbersep=10pt,
    captionpos=t,
    escapeinside={\%*}{*)}
}

%%%----------%%%----------%%%----------%%%----------%%%

\begin{document}

\title{Python}

\author{Aditya Kumar}

\date{01/08/2025}

\maketitle
	
	% start here
\section{Basic}

\begin{itemize}
	\item Python was created by Guido van Rossum in the early 90s. It is now one of the most popular languages in existence. I fell in love with Python for its syntactic clarity. It's basically executable pseudocode.
	\item Single line comments start with \# (numbered symbol)
	
	\begin{lstlisting}
		# this is your single line comment.
	\end{lstlisting}
	
	\item Multiline strings can be written
	using three "s, and are often used
	as documentation.
	
	\begin{lstlisting}
		""" this
			is
			your
			multi-line
			string
		"""
		# it can be used in printing ascii arts
		# for example :
		art="""
		   _____                .__.__                 __   
		/  _  \   ______ ____ |__|__| _____ ________/  |_ 
		/  /_\  \ /  ___// ___\|  |  | \__  \\_  __ \   __\
		/    |    \\___ \\  \___|  |  |  / __ \|  | \/|  |  
		\____|__  /____  >\___  >__|__| (____  /__|   |__|  
		\/     \/     \/             \/             
		"""
		# printing art
		print(art)
	\end{lstlisting}
	\subsection{Primitive Datatypes and Operators}
	\item 
\end{itemize}


\end{document}
