
\documentclass[a4paper,11pt]{article}

\usepackage[T1]{fontenc}
\usepackage[utf8]{inputenc}
\usepackage{graphicx}
\usepackage{xcolor}

\renewcommand\familydefault{\sfdefault}
\usepackage{tgheros}
\usepackage[defaultmono]{droidsansmono}

\usepackage{amsmath,amssymb,amsthm,textcomp}
\usepackage{enumerate}
\usepackage{multicol}
\usepackage{tikz}

\usepackage{geometry}
\geometry{left=25mm,right=25mm,%
bindingoffset=0mm, top=20mm,bottom=20mm}


\linespread{1.3}

\newcommand{\linia}{\rule{\linewidth}{0.5pt}}

% custom theorems if needed
\newtheoremstyle{mytheor}
    {1ex}{1ex}{\normalfont}{0pt}{\scshape}{.}{1ex}
    {{\thmname{#1 }}{\thmnumber{#2}}{\thmnote{ (#3)}}}

\theoremstyle{mytheor}
\newtheorem{defi}{Definition}

% my own titles
\makeatletter
\renewcommand{\maketitle}{
\begin{center}
\vspace{2ex}
{\huge \textsc{\@title}}
\vspace{1ex}
\\
\linia\\
\@author \hfill \@date
\vspace{4ex}
\end{center}
}
\makeatother
%%%

% custom footers and headers
\usepackage{fancyhdr}
\pagestyle{fancy}
\lhead{}
\chead{}
\rhead{}
%\lfoot{Assignment \textnumero{} 5}
\cfoot{}
%\rfoot{Page \thepage}
\renewcommand{\headrulewidth}{0pt}
\renewcommand{\footrulewidth}{0pt}
%

% code listing settings
\usepackage{listings}
\lstset{
    language=Python,
    basicstyle=\ttfamily\small,
    aboveskip={1.0\baselineskip},
    belowskip={1.0\baselineskip},
    columns=fixed,
    extendedchars=true,
    breaklines=true,
    tabsize=4,
    prebreak=\raisebox{0ex}[0ex][0ex]{\ensuremath{\hookleftarrow}},
    frame=lines,
    showtabs=false,
    showspaces=false,
    showstringspaces=false,
    keywordstyle=\color[rgb]{0.627,0.126,0.941},
    commentstyle=\color[rgb]{0.133,0.545,0.133},
    stringstyle=\color[rgb]{01,0,0},
    numbers=left,
    numberstyle=\small,
    stepnumber=1,
    numbersep=10pt,
    captionpos=t,
    escapeinside={\%*}{*)}
}

%%%----------%%%----------%%%----------%%%----------%%%

\begin{document}

\title{Python}

\author{Aditya Kumar}

\date{01/08/2025}

\maketitle
	
	% start here
\section{Basic}

\begin{itemize}
	\item Python was created by Guido van Rossum in the early 90s. It is now one of the most popular languages in existence. I fell in love with Python for its syntactic clarity. It's basically executable pseudocode.
	\item Single line comments start with \# (numbered symbol)
	
	\subsection{Comments}
	\begin{lstlisting}
		# this is your single line comment.
	\end{lstlisting}
	
	\item Multiline strings can be written
	using three "s, and are often used
	as documentation.
	
	\begin{lstlisting}
		""" this
			is
			your
			multi-line
			string
		"""
		# it can be used in printing ascii arts
		# for example :
		art="""
		   _____                .__.__                 __   
		/  _  \   ______ ____ |__|__| _____ ________/  |_ 
		/  /_\  \ /  ___// ___\|  |  | \__  \\_  __ \   __\
		/    |    \\___ \\  \___|  |  |  / __ \|  | \/|  |  
		\____|__  /____  >\___  >__|__| (____  /__|   |__|  
		\/     \/     \/             \/             
		"""
		# printing art
		print(art)
	\end{lstlisting}
	\subsection{helloWorld}
	\item We can print any string with print("String")
	\item Here print() is a function.
	\item Anything between "" is String.
	\begin{lstlisting}
		print("hello world")
	\end{lstlisting}
	\item This will give output \textit{hello World} on the console
	
	\subsection{Escape Sequence}
	\item In python the escape sequence represent new line character
	\item We can print multiple line in shell using only one line of code.
	\item we use \textit{\textbackslash n} where we want to break the line
	\begin{lstlisting}
		print("This is first line.\nThis is second line.\nThis is third line.\n and goes on.........")
	\end{lstlisting}
	\item Previous command will give us the following output
	\begin{lstlisting}
		This is first line.
		This is second line.
		This is third line.
		 and goes on.........
	\end{lstlisting}
	\subsection{Concatenate}
	\item Concatenate means joining two or more things together in sequence.
	
	\item It usually refers to joining strings, arrays, or lists end-to-end.
	
	\item The result of concatenation is a single combined object of the same type (string + string = string, list + list = list).
	
	\item Concatenation is order-sensitive: "A" + "B" is different from "B" + "A".
	
	\item Only compatible types can be concatenated (string + string works, string + number does not unless converted).
	
	\item Concatenation does not alter the original objects unless you explicitly reassign the result.
	\begin{lstlisting}
		# String + String -> String
		a = "Hello"
		b = "World"
		result = a + b
		print(result)
		print(type(result))
		
		# Output
		HelloWorld
		<class 'str'>
		
		
		
		# List + List -> List
		list1 = [1, 2, 3]
		list2 = [4, 5, 6]
		result = list1 + list2
		print(result)
		print(type(result))
		
		
		# Output
		[1, 2, 3, 4, 5, 6]
		<class 'list'>
		
	\end{lstlisting}
	
	\subsection{type()}
	\item \textit{type()} is a built-in Python function used to find the type (class) of an object.
	
	\item It tells what kind of data (string, list, int, float, etc.) the object belongs to. [we will know about these in next section]
	
	\item Useful for debugging, validation, and understanding how Python treats different values.
	
	\item The return value of type() itself is a type object (like <class 'str'>, <class 'list'>).
	
	\item Objects of the same type can usually be combined, manipulated, or iterated in similar ways.
	
	\item For String ;
	\begin{lstlisting}
		# for String
		
		text = "Python"
		print("text = " , text , type(text))
		
		
		# output should be like
		text =  Python <class 'str'>
	\end{lstlisting}
	\item For Int ;
	\begin{lstlisting}
		# for int
		num = 42
		print("num = " , num , type(num))
		
		
		# output
		num =  42 <class 'int'>
		
	\end{lstlisting}
	\item For list ;
	\begin{lstlisting}
		# for list
		numbers = [1, 2, 3]
		print("list = " , numbers , type(numbers))
		
		
		# output
		list =  [1, 2, 3] <class 'list'>
	\end{lstlisting}
	
	\subsection{input()}
	\item input() is a built-in Python function used to take user input from the keyboard.
	
	\item It always returns the entered data as a string, no matter what the user types.
	
	\item If you need another type (like int or float), you must convert the result using functions like int() or float().
	
	\item It pauses program execution until the user presses Enter.
	
	\item Useful for interactive programs where user input is required.
	
	\item String ;
	\begin{lstlisting}
		# string 
		name = input("Enter your name: ")
		print("Hello,", name)
		print(type(name))
		
		# output
		Enter your name: roxx
		Hello, roxx
		<class 'str'>
		
	\end{lstlisting}
	
	\item Converting Inputing to Integers ;
	\begin{lstlisting}
		age = int(input("Enter your age: "))
		print("You are", age, "years old")
		print(type(age))
		
		
		# output
		Enter your age: 25
		You are 25 years old
		<class 'int'>
		
	\end{lstlisting}
	
	\item Converting input to float ;
	\begin{lstlisting}
		pi_val = float(input("Enter value of pi: "))
		print("Pi is approx:", pi_val)
		print(type(pi_val))
		
		
		# output 
		Enter value of pi: 3.147
		Pi is approx: 3.147
		<class 'float'>
		
	\end{lstlisting}
	
	\subsection{Typecasting}
	\item \textit{Typecasting} means converting one data type into another in Python.
	
	\item It is done using constructor functions like int(), float(), str(), list(), tuple(), etc.
	
	\item Typecasting is required when you want to perform operations that need specific types (e.g., arithmetic on numbers instead of strings).
	
	\item Some conversions are safe and natural (e.g., int("10") → 10), while others may raise errors (e.g., int("abc") → error).
	
	\item Typecasting always creates a new object of the target type; the original object remains unchanged.
	
	\item String -> Integers
	\begin{lstlisting}
		# String -> Integers
		s = "123"
		num = int(s)
		print(num, type(num))
		
		
		# output
		123 <class 'int'>
	\end{lstlisting}
	
	\item String -> float
	\begin{lstlisting}
		# String -> float
		s = "3.14"
		pi = float(s)
		print(pi, type(pi))
		
		# output
		3.14 <class 'float'>
	\end{lstlisting}
	
	\item Integer -> string
	\begin{lstlisting}
		#Integer -> String
		n = 42
		s = str(n)
		print(s, type(s))
		
		
		# output
		# 42 <class 'str'>
	\end{lstlisting}
	
	\item list -> tuple
	\begin{lstlisting}
		# List -> tuple
		
		lst = [1, 2, 3]
		t = tuple(lst)
		print(t, type(t))
		
		# output
		# (1, 2, 3) <class 'tuple'>
	\end{lstlisting}
	
	\item tuple -> list
	\begin{lstlisting}
		# tuple -> list
		
		t = (4, 5, 6)
		lst = list(t)
		print(lst, type(lst))
		
		
		# output
		# [4, 5, 6] <class 'list'>
	\end{lstlisting}
	
	\subsection{len() function}
	\item len() is a built-in Python function that returns the number of items in an object.
	
	\item It works on sequences (strings, lists, tuples) and collections (sets, dictionaries).
	
	\item For strings, len() counts the number of characters.
	
	\item For lists, tuples, and sets, len() counts the number of elements.
	
	\item For dictionaries, len() counts the number of key-value pairs.
	
	\item It does not work on integers or floats directly (unsupported types).
	
	\item String ;
	\begin{lstlisting}
		a = " : "
		# String
		text = "Python"
		print(text , len(text))
		
		# output
		# Python 6
	\end{lstlisting}
	
	\item list ;
	\begin{lstlisting}
		# list
		nums = [10, 20, 30, 40]
		print(nums ,a, len(nums))
		
		# output
		# [10, 20, 30, 40]  :  4
	\end{lstlisting}
	\item tuple ;
	\begin{lstlisting}
		# tuple ;
		t = (1, 2, 3, 4, 5)
		print(t,a,len(t))
		
		
		# output
		# (1, 2, 3, 4, 5)  :  5
		
	\end{lstlisting}
	\item set ;
	\begin{lstlisting}
		# set ;
		s = {1, 2, 3, 4}
		print(s , a, len(s))
		
		# output
		# {1, 2, 3, 4}  :  4
		
	\end{lstlisting}
	
\end{itemize}


\end{document}
