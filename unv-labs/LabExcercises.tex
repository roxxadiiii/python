%%% Template originaly created by Karol Kozioł (mail@karol-koziol.net) and modified for ShareLaTeX use

\documentclass[a4paper,11pt]{article}

\usepackage[T1]{fontenc}
\usepackage[utf8]{inputenc}
\usepackage{graphicx}
\usepackage{xcolor}

\renewcommand\familydefault{\sfdefault}
\usepackage{tgheros}
\usepackage[defaultmono]{droidsansmono}

\usepackage{amsmath,amssymb,amsthm,textcomp}
\usepackage{enumerate}
\usepackage{multicol}
\usepackage{tikz}

\usepackage{geometry}
\geometry{left=25mm,right=25mm,%
bindingoffset=0mm, top=20mm,bottom=20mm}


\linespread{1.3}

\newcommand{\linia}{\rule{\linewidth}{0.5pt}}

% custom theorems if needed
\newtheoremstyle{mytheor}
    {1ex}{1ex}{\normalfont}{0pt}{\scshape}{.}{1ex}
    {{\thmname{#1 }}{\thmnumber{#2}}{\thmnote{ (#3)}}}

\theoremstyle{mytheor}
\newtheorem{defi}{Definition}

% my own titles
\makeatletter
\renewcommand{\maketitle}{
\begin{center}
\vspace{2ex}
{\huge \textsc{\@title}}
\vspace{1ex}
\\
\linia\\
\@author \hfill \@date
\vspace{4ex}
\end{center}
}
\makeatother
%%%

% custom footers and headers
\usepackage{fancyhdr}
\pagestyle{fancy}
\lhead{}
\chead{}
\rhead{}
\lfoot{Python Questions}
\cfoot{}
\rfoot{Page \thepage}
\renewcommand{\headrulewidth}{0pt}
\renewcommand{\footrulewidth}{0pt}
%

% code listing settings
\usepackage{listings}
\lstset{
    language=Python,
    basicstyle=\ttfamily\small,
    aboveskip={1.0\baselineskip},
    belowskip={1.0\baselineskip},
    columns=fixed,
    extendedchars=true,
    breaklines=true,
    tabsize=4,
    prebreak=\raisebox{0ex}[0ex][0ex]{\ensuremath{\hookleftarrow}},
    frame=lines,
    showtabs=false,
    showspaces=false,
    showstringspaces=false,
    keywordstyle=\color[rgb]{0.627,0.126,0.941},
    commentstyle=\color[rgb]{0.133,0.545,0.133},
    stringstyle=\color[rgb]{01,0,0},
    numbers=left,
    numberstyle=\small,
    stepnumber=1,
    numbersep=10pt,
    captionpos=t,
    escapeinside={\%*}{*)}
}

%%%----------%%%----------%%%----------%%%----------%%%

\begin{document}

\title{Python Lab Excercises}

\maketitle
\begin{enumerate}
\section{Basic Syntax:}
%\begin{enumerate}
	\item Write a programs that takes two numbers and prints them on the screen .
	\item Write a program that prompts the user for their name and prints a welcome message to them .
	\item Write a program that prints the squares of all the numbers less than or equal to a given number .
	\item Write a program for swapping the values of two variables without using a third variable .
	\item Write a program for swapping the values of two variables without using a third variable .
	\item Write a program that finds GCD of two given numbers .
	\item Write a python function SumOfList that takes a list of integers and return their sum .
%\end{enumerate}
\section{Control Structures:}
%\begin{enumerate}
	\item Write a python program that checks whether a given number is even or odd .
	\item Write a python prorgam that find the largest among three numbers provoded by the user .
	\item Write a prorgam that finds the largest number from the given numbers .
%\end{enumerate}
\section{Loops:}
%\begin{enumerate}
	\item Write a Python program that prints all the numbers between a given range.
	\item Write a python program that prints all the even number between a given range.
	\item Write a python program that prints all the odd numbers between a given range.
	\item Write a python program that prints a table of a given number.
	\item Write a python program to calculate the factorial of a number using a for loop.
	\item write a python program to find if a given number is prime or not.
	\item write a python program to find the mean, median,mode,variance and standard deviation of a given set of numbers.
	\item write a python program to print the fibonacci sequesnce up to n terms, where n  is the input by the user.
	\item write a python program to display the sum of all the given number without using any string functions.
	\item write a program that generates and returns all the prime factors of a given number.
	\item write a program that displays the weights of all the digits in a given number without using any library functions
\section{Strings:}
	\item Write a python program that counts the number of vowels and consonants in a given string.
	\item write a python functions that counts the numbers of words in a string.
	\item write a python functions that counts the numbers of lines in a given string.
	\item write a python functions that prints the numbers of digits and punctuations marks that appear in a string.
	\item write a python program to reverse a given string without using slicing.
	
\section{Lists:}
	\item Write a python program that prints all the numbers in a list stored on a given range of indexes.
	\item write a python program to remove all duplicates from a given list.
	\item write a python functions FindMaxMin that returns the maximum values in a list.
\section{Dictonaries:}
	\item write a python programs to count the frequency of each word in a given text.
	\item write a python function that takes a dictonary as input and returns the key with the highest value.
\section{Functions and Arguments:}
	\item Write a python function isPrime that checks if a number is prime number.
	\item write a python function that accepts an arbitrary number of keyword arguments and prints the result as a dictonary.
	\item write a function that takes a list of numbers and returns the square of all those numbers.
	\item write a function that takes the radius of a circle and returns the diameter,area,and circumference of it.
	\item write a function that takes a number as input and tells if it is divisible by 3 or not,without dividing it.
	\item write a function that takes a number as input and tells if it is divisible by 11 or not,without dividing it.
\section{List Comprehensions:}
	\item write a python program that takes a list of integers and return a new list that contains only the even numbers using list comprehensions.
	\item write a python list comprehension to generate a list of squares of number between 1 and 10.

\section{File Handeling:}
	\item Write a python program to read a text file and print the numbers of lines,words,and characters in it.
	\item write a python program to write the contents of one file into another file.
	\item write a program that finds a given string in a given file.
	\item write a program that finds a given string in a file and returns the place where the string appears in the file.your prorgram should return the index of the first character of the given string in the file.
	\item write a python program that finds a given word in a file and replaces all the instances of the given word with another given word(similiar to find and replace all functionality).
	
\section{Exception Handling:}
	\item Write a python program that handles the ZeroDivisionError exception.
	\item write a python function that raises a ValueError if a non-numeric values is passed as an argument
	
\section{Classes and Objects:}
	\item Write a Python class Rectangle that has attributes for length and has attributes for length and width and methods to calculate the area and perimeter.
	\item Write a python class BankAccount with methods to deposit,withdraw,and check the balance of the account.
	\item Write a Python class with a method that behaves differently based on the number of arguments passed (simulate method overloading).
	
	\item Implement a class with an add method that can add two or three numbers using method overloading techniques .
	
	\item Create a method that performs different operations based on argument types (e.g. Multiply if both are ints , repeat if first is string and second is int).
\end{enumerate}
\end{document}
